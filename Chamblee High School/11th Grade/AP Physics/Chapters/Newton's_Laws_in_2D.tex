% Newton's Laws in 2D
\section{Newton's Laws in 2D}
	
% Newton's Laws in 2D -  Newton's Third Law
\subsection{Newton's $2^{nd}$ Law in 2-D}
The situation is more complicated when forces act in more than one dimension. You must still identify all forces and draw your force diagram. You then resolve your problem into an x-problem and a y-problem.
	
% Newton's Laws in 2D -  Mass and Weight
\subsection{Mass and Weight}
Many people think \textbf{mass} and \textbf{weight} are the same thing. They are not. Mass is \textbf{inertia}, or resistance to acceleration. Weight can be defined as the \textbf{force due to gravitation attraction} \[W=mg\]
	
% Newton's Laws in 2D - Apparent Weight
\subsection{Apparent Weight}
If an object subject to gravity is not in free fall, then there must be a \textbf{reaction force} to act in opposition to gravity. We sometimes refer to this reaction force as \textbf{apparent weight}.
	
% Newton's Laws in 2D - Elevator Rides
\subsection{Elevator Rides}
When you are in an elevation, \textbf{your actual weight ($mg$) never changes}. You feel lighter or heavier during the ride because your \textbf{apparent weight} increases when you are accelerating up, decreases when you are accelerating down, and is equal to your weight when you are not accelerating at all.
	
% Newton's Laws in 2D - Normal Force
\subsection{Normal Force}
The \textbf{normal force} is a force that keeps one object from penetrating into another object. the normal force is always \textbf{perpendicular} to a surface. The normal exactly cancels out the components of all applied forces that are perpendicular to a surface. A normal force can also exist that is totally unrelated to the weight of an object.
	
% Newton's Laws in 2D - Normal Force on a flat surface
\subsection{Normal Force on a flat surface}
The normal force is equal to the weight of an object for objects resting on a horizontal surface. \[N=W=mg\]
	
% Newton's Laws in 2D - Normal Force on a ramp
\subsection{Normal Force on a ramp}
The normal force is perpendicular to angled ramps as well. It's always equal to the component of weight perpendicular to the surface. \[N=m*g*\cos\theta\]