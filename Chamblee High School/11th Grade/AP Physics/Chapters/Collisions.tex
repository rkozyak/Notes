% Collisions
\section{Collisions}

% Overview
\subsection{Overview}
When two moving objects make contact with each other, they undergo a \textbf{collision}. Conservation of momentum is used to analyze all collisions. Newton's Third law is also useful. It tells us that a force exerted by body A on body B in a collision is equal and opposite to the force exerted on body B by body A. During a collision, external forces are ignored. The time frame of the collision is also very short and the forces are impulsive forces.

% Collision Types
\subsection{Collision Types}
\begin{itemize}
	\item \textbf{Elastic Collisions} - No deformation occurs and no kinetic energy is lost
	\item \textbf{Inelastic Collisions} - Deformation occurs and kinetic energy is lost
	\item \textbf{Perfectly Inelastic Collisions} - Objects stick together and become one object. Deformation also occurs and kinetic energy is lost
\end{itemize}

% (Perfectly) Inelastic Collisions
\subsection{(Perfectly) Inelastic Collisions}
This is the simplest type of collision. After the collision, there is only one velocity, since there is only one object. Kinetic energy is also lost. Explosions are the reverse of perfectly inelastic collisions in which kinetic energy is gained. 

% Elastic Collisions
\subsection{Elastic Collisions}
After the collision, there are still two objects but with two separate velocities. Kinetic energy remains constant before and after the collision and therefore, two basic equations must hold true for all elastic collision: \[\sum p_b =\sum p_a\] \[\sum K_b =\sum K_a\]

% 2D Collisions
\subsection{2D Collisions}
Momentum in the x-direction is conserved:
\[\sum P_x\mbox{ (before) }= \sum P_x\mbox{ (after) }\]

Momentum in the y-direction is conserved:
\[\sum P_y\mbox{ (before) }= \sum P_y\mbox{ (after) }\] 

You must treat x and y coordinates independently. You ignore x when calculating y and ignore y when calculating x.