% Conservation of Energy
\section{Conservation of Energy}

% Law of Conservation of Energy
\subsection{Law of Conservation of Energy}
In any isolated system, the total energy remains constant. Energy can neither be created nor destroyed, but can only be transformed from one type of energy to another. 

% Conservation of Mechanical Energy
\subsection{Law of Conservation of Mechanical Energy}
\begin{itemize}
	\item E = K+U = Constant
	\begin{itemize}
		\item K = Kinetic Energy ($\frac{1}{2}mv^2$)
		\item U = Potential Energy (gravity or spring)
	\end{itemize}
	
	\item $\Delta$E = $\Delta$U + $\Delta$K = 0
		\begin{itemize}
		\item $\Delta$K = Change in kinetic energy
		\item $\Delta$U = Change in gravitational or spring potential energy
	\end{itemize}
\end{itemize}

% Pendulums and Energy Conservation
\subsection{Pendulums and Energy Conservation}
Energy goes back and forth between K and U. At the highest point, all energy is U. As it drops, U goes to K. At the bottom, energy is all K.

% Springs and Energy Conservation
\subsection{Springs and Energy Conservation}
Transforms energy back and forth between K and U. When fully stretched or extended, all energy is U. When passing through equilibrium, all its energy is K. At other points in its cycle, the energy is a mixture of U and K.

% Law of Conservation of Energy Formula
\subsection{Law of Conservation of Energy Formula}
\begin{itemize}
	\item E = U + k+ E$\textsubscript{initial}$ = Constant
	\item $\Delta$U + $\Delta$K + $\Delta$E$\textsubscript{initial}$ = 0
	\item E$\textsubscript{inital}$ is thermal energy.
	\item Mechanical energy may be converted to and from heat.
\end{itemize}

% Work done by non-conservative forces
\subsection{Work done by non-conservative forces}
\begin{itemize}
	\item $W\textsubscript{net}=W_c+W_{nc}$
	\begin{itemize}
		\item Net work is done by conservative and non-conservative forces
		\item $W_c = -\Delta U$
		\item $W\textsubscript{net}=\Delta K$
		\item $\Delta K = -\Delta U + W_{nc}$
	\end{itemize}
	\item $W_{nc}=\Delta U + \Delta K = \Delta E$
	\item Non-conservative forces change mechanical energy. If non-conservative work is negative, as it often is, the mechanical energy of the system will drop.
\end{itemize}