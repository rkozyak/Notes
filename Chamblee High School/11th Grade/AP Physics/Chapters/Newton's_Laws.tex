% Newton's Laws
\section{Newton's Laws}
	
% Newton's Laws - Isaac Newton
\subsection{Isaac Newton}
Isaac Newton is arguably the greatest scientific genius ever. He came up with 3 Laws of Motion to explain the observations and analyses of Galileo and Johannes Kepler. He also discovered that white light was composed of many colors all mixed together. He invented new mathematical techniques such as calculus and binomial expansion theorem in his study of physics. He published his laws in 1687 in the book Mathematical Principles of Natural Philosophy.
	
% Newton's Laws - What is Force?
\subsection{What is Force?}
A force is a \textbf{push} or a \textbf{pull} on an object. Forces cause an object to \textbf{accelerate}.
	
% Newton's Laws - Newton's First Law
\subsection{Newton's First Law}
\textbf{A body in motion stays in motion at a constant velocity} and a body at rest stays at rest \textbf{unless acted upon by a net external force}. This law is commonly referred to as the \textbf{Law of Inertia}. If there is \textbf{zero net force} on a body, it cannot accelerate, and therefore must move at a constant velocity. 
	
% Newton's Laws - Mass and Inertia
\subsection{Mass and Inertia}
Chemists like to define mass as the amount of "stuff" or "matter" a substance has. Physicists define \textbf{mass} as \textbf{inertia}, which is the ability of a body to resit acceleration by a net force.
	
% Newton's Laws - Newton's Second Law
\subsection{Newton's Second Law}
A body \textbf{accelerates} when acted upon by a \textbf{net external force}. This acceleration is proportional to the net force and is in the direction which the net force acts. \[\sum F=ma\]

\begin{itemize}
	\item where $\sum F$ is the net force measured in Newtons ($N$)
	\item $m$ is mass (kg)
	\item $a$ is acceleration ($m/s^2$)
\end{itemize}
	
% Newton's Laws - Units of Force
\subsection{Units of Force}
\begin{itemize}
	\item Newton (SI System)
	\item $1N = 1kg*m/s^2$
	\item 1$N$ is the force required to accelerate a 1$kg$ mass at a rate if 1 $m/s^2$
\end{itemize}
	
% Newton's Laws - Newton's Third Law
\subsection{Newton's Third Law}
For every action there exist an \textbf{equal and opposite reaction}.