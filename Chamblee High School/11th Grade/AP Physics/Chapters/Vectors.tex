%Vectors
\section{Vectors}
	
% Overview
\subsection{Overview}
There are two kinds of quantities...
\begin{itemize}
	\item Scalars are the quantities that have \textbf{magnitude} only
	\begin{itemize}
		\item position
		\item speed
		\item time
		\item mass
	\end{itemize}
	
	\item Vectors are quantities that have both \textbf{magnitude} and \textbf{direction}, such as
	\begin{itemize}
		\item displacement
		\item velocity
		\item acceleration
	\end{itemize}
\end{itemize}
	
% Direction of Vectors
\subsection{Direction of Vectors}
Vector \textbf{direction} is the direction of the arrow, given by an \textbf{angle}.
	
% Magnitude of Vectors
\subsection{Magnitude of Vectors}
The best wya to determine the \textbf{magnitude} of a vector is to measure its \textbf{length}. The length of the vector \textit{proportional} to the magnitude (or size) of the quantity it represents.
	
% qual and Inverse Vectors
\subsection{Equal and Inverse Vectors}
\textbf{Equal vectors} have the \textbf{same length and direction}, and represent the same quantity (such as force or velocity).
	
\textbf{Inverse vectors} have the \textbf{same length}, but \textbf{opposite direction}.
	
% Vectors: x-component
\subsection{Vectors: x-component}
The x-component of a vector is the "shadow" it cases on the x-axis. 
\[cos \theta = \frac{\text{adjacent}}{\text{hypotenuse}}\]
	
% Vectors: y-component
\subsection{Vectors: y-component}
The y-component of a vector is the "shadow" it casts on the y-axis. 
\[sin\theta=\frac{\text{opposite}}{\text{hypotenuse}}\]
	
% Vectors: Angle
\subsection{Vectors: angle}
The angle of a vector makes with the x-axis can be determined by the components. It is calculated by the inverse tangent function
\[\theta = \tan^{-1}\frac{\text{opposite}}{\text{adjacent}}\]
	
% Vectors: magnitude
\subsection{Vectors: magnitude}
The magnitude of a vector can be determined by the components. It is calculated by using the Pythagorean Theorem. \[A^2+B^2=C^2\]