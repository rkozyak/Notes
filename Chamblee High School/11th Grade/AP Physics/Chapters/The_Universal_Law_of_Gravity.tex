% The Universal Law of Gravity
\section{The Universal Law of Gravity}

% Overview
\subsection{Overview}
The Universal Law of Gravity tells us that the force of gravity objects exert on each other depends on their masses and the distance they are separated from each other. You used the formula $F_g=mg$ to approximate the force of gravity on an object near the earth's surface. This formula wont work for planets, space travel or for objects that are far from earth.

% The Force of Gravity
\subsection{The Force of Gravity}
The Universal Law of gravity \textbf{always} works, whereas $F_g=mg$ only works sometimes. \[F_g=\frac{Gm_1m_2}{r^2}\]
\begin{itemize}
	\item $F_g$ = Force due to gravity ($N$)
	\item $G$ = Universal gravitational constant $6.67*10^-11\frac{Nm^2}{kg^2}$
	\item $m_1$ and $m_2$ = the two masses ($kg$)
	\item $r$ = the distance between the cents of the masses ($m$)
\end{itemize}

% Acceleration due to Gravity
\subsection{Acceleration due to Gravity}
$g=9.8m/s^2$ works fine when we are near the surface of the earth, For space travel, we need a different formula. \[g=\frac{GM}{r^2}\] 

This formula lets you calculate $g$ anywhere if you know the distance a body is from the center of a planet and you can use it to calculate the acceleration to gravity anywhere.