% Kinematics
\section{Introduction to Work}

% Overview
\subsection{Overview}
Work tells us how much a force or combination of forces changes the energy of a system. Work is the bridge factor between force (a vector) and energy (a scalar).
\[W=F(\Delta x)\cos\theta\]
Where:
\begin{itemize}
	\item $F$ = Force ($N$)
	\item $\Delta x$ = Displacement ($m$)
	\item $\theta$ = Angle between force and displacement
\end{itemize}

Forces and direction of motion also both matter in defining work. There is no work done by a force if it causes no displacement. Forces can do positive, negative, or zero work. 

% Units of Work
\subsection{Units of Work}
\begin{itemize}
	\item SI System: Joule ($Nm$)
	\item British System: foot-pound
	\item cgs System: erg (dyne-cm)
	\item Atomic Level: electron-Volt (eV)
\end{itemize}

% Work and Energy
\subsection{Work and Energy}
Work changes mechanical energy! If an applied force does positive work on a system, it tries to increase mechanical energy. If an applied force does negative work, it tries to decrease mechanical energy. The two forms of mechanical energy are called potential and kinetic energy.

% Kinetic Energy
\subsection{Kinetic Energy}
Kinetic Energy is energy due to motion. \[K=\frac{1}{2}mv^2\] Where:
\begin{itemize}
	\item $K$ = Kinetic Energy
	\item $m$ = mass in $kg$
	\item $v$ = speed in $m/s$
\end{itemize}

% The Work-Energy Theorem
\subsection{The Work-Energy Theorem}
The Work-Energy Theorem states that the net work due to all forces equals the change in the kinetic energy of a system. \[W\textsubscript{net}=\Delta K\] Where:
\begin{itemize}
	\item $W\textsubscript{net}$ = work due to all forces acting on an object
	\item $\Delta K$ = change in kinetic energy ($K_f-K_i$)
\end{itemize}

% Springs
\subsection{Springs}
When a spring is stretched or compressed from it equilibrium position, it does \textbf{negative work}, since the spring pulls opposite the direction of motion. \[W_s=-\frac{1}{2}kx^2\] Where:
\begin{itemize}
	\item $W_s$ = work done by spring $J$
	\item $k$ = force constant of spring ($N/m$)
	\item $x$ = displacement from equilibrium ($m$)
\end{itemize}
The force doing the stretching does \textbf{positive work} equal to the magnitude of the work done by the spring. \[W\textsubscript{applied}=-\frac{1}{2}kx^2\]

% Spring Tension
\subsection{Tension}
Tension is a \textbf{pulling force} that arises when a \textbf{rope, string, or other long thin material resists being pulled apart} without stretching significantly. Tension always pulls away from a body attached to a rope or string and toward the center of the rope or string. 
	
% Springs (Hooke's Law)
\subsection{Springs (Hooke's Law)}
The magnitude of the force exerted by a spring is proportional to the amount it is stretched. \[F=-kx\] 
\begin{itemize}
	\item $F$ = force exerted by the spring ($N$)
	\item $k$ = force constant of the Spring ($N/m$ or $N/cm$)
	\item $x$ = displacement from equilibrium position ($m$ or $cm$)
\end{itemize}

% Power
\subsection{Power}
Power is the rate of which work is done. \[P=\frac{W}{\Delta t}=\frac{F\Delta x}{\Delta t} = Fv\] Where:
\begin{itemize}
	\item $W$ = work in Joules
	\item $\Delta t$ = elapses time in seconds.
\end{itemize}

% Unit of Power
\subsection{Unit of Power}
The SI unit for Power is the Watt It is named after the Scottish engineer James Watt (1776-1819) who perfected the steam engine. \[1 Watt = 1 Joule/s\] The kilowatt-hour is a commonly used unit by electrical power companies. Power companies charge you by the kilowatt-hour ($kWh$) but this is not power, it is really energy consumed. \[1kW = 1000W=1000J/s*3600s=3.6*10^6J\]

% Forces Types
\subsection{Forces Types}
Forces acting on a system can be divided into two types according to how they affect potential energy. \\\textbf{Conservative forces} can be related to potential energy changes. \\\textbf{Non-conservative forces} cannot be related to potential energy changes.

% Conservative Forces
\subsection{Conservative Forces}
In conservative forces:
\begin{itemize}
	\item Work is \textbf{path independent}. Work can be calculated from the starting and ending points only. The actual path is ignored in calculations.
	\item \textbf{Work along a closed path is zero}. If the starting and ending points are the same, no work is done by the force.
	\item \textbf{Work changes potential energy}.
	\item \textbf{Conservation of mechanical energy holds!}
\end{itemize}

% Non-Conservative Forces
\subsection{Non-Conservative Forces}
In non-conservative forces:
\begin{itemize}
	\item Work is \textbf{path dependent}. Knowing the starting and ending points is not sufficient to calculate the work.
	\item \textbf{Work along a closed path is NOT zero}.
	\item \textbf{Work changes mechanical energy}.
	\item \textbf{Conservation of mechanical energy does not hold!}
\end{itemize}

% Potential Energy
\subsection{Potential Energy}
Potential Energy is the energy of position or configuration. It is also know as "stored" energy. 
\begin{itemize}
	\item For gravity: $U_g=mgh$
	\begin{itemize}
		\item $m$ = mass
		\item $g$ = acceleration due to gravity
		\item $h$ = height above the "zero" point
	\end{itemize}
	
	\item For springs: $U_s=\frac{1}{2}kx^2$
	\begin{itemize}
		\item $k$ = spring force constant
		\item $x$ = displacement from equilibrium position
	\end{itemize}
\end{itemize}

% Conservative forces and Potential energy
\subsection{Conservative forces and Potential energy}
\[W_c=-\Delta U\] If a conservative force does positive work on a system, potential energy is lost. If a conservative force does negative work , potential energy is gained.
\begin{itemize}
	\item For Gravity: $-W_g=-\Delta U_g=-({mgh}_f-{mgh}_i)$
	\item For Springs: $-W_s=-\Delta U=-(\frac{1}{2}kx_f^2-\frac{1}{2}kx_i^2)$
\end{itemize}