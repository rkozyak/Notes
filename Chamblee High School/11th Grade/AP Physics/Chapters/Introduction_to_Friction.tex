% Introduction to Friction
\section{Introduction to Friction}
	
% Introduction to Friction - Overview
\subsection{Overview}
Friction is the force that opposes a sliding motion. Friction is due to microscopic irregularities in even the smoothest of surfaces. Friction is highly useful and enables us to walk and drive a car, among other things. Friction is also dissipative. That means it causes mechanical energy to be converted to head. The friction that exits between two surfaces is directly proportional to the normal force.
	
% Introduction to Friction - Static Friction
\subsection{Static Friction}
Static friction occurs between two surfaces that are not slipping relative to each other. \[f_s\leq\mu_sN\]
\begin{itemize}
	\item $f_s$ =  Static frictional force ($N$)
	\item $\mu_s$ = coefficient of static friction
	\item $N$ = normal force (N)
\end{itemize}
Static frictions between two surfaces is \textbf{zero unless there is a force} trying to make the surfaces slide on one another. Static friction can \textbf{increase} as the force trying to push an object \textbf{until it reaches its maximum allowed value as defined by $\mu_s$} Once the maximum value of static friction has been exceeded by an applied force, the surfaces begin to slide and the friction is no longer static friction.
	
% Introduction to Friction - Kinetic Friction
\subsection{Kinetic Friction}
Kinetic friction \textbf{occurs between surfaces that are slipping past each other}. \[f_k=\mu_kN\]
\begin{itemize}
	\item $f_k$ = kinetic frictional force ($N$)
	\item $\mu_k$ = coefficient of kinetic friction
	\item $N$ = normal force ($N$)
\end{itemize}
Kinetic friction (sliding friction) is \textbf{generally less than static friction} (motionless friction) for most surfaces.