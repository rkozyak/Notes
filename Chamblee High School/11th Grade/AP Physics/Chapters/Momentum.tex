% Momentum
\section{Momentum}

% Overview
\subsection{Overview}
\textbf{Momentum is a measure of how hard it is to stop or turn a moving object.} Momentum is related to both mass and velocity. Momentum is possessed by all moving objects.

% Calculating Momentum
\subsection{Calculating Momentum}
\begin{itemize}
	\item For one particle: $p=mv$ 
	\item For a system of multiple particles: $P=\sum P_i=\sum m_iv_i$\
	\item Momentum is a vector with the same direction as the velocity vector
	\item The unit of Momentum is $kg*m/s$ or $Ns$
\end{itemize}

% Change in momentum
\subsection{Change in Momentum}
Like any change, change is momentum is calculated by looking at final and initial momentum. \[\Delta p = p_f-p_i\]
\begin{itemize}
	\item $\Delta p$ = Change in momentum
	\item $p_f$ = Final momentum
	\item $p_i$ = initial momentum
\end{itemize}

% Impulse
\subsection{Impulse}
\textbf{Impulse} is the \textbf{product of} an external \textbf{force and time}, which results in a \textbf{change in momentum} of a particle or system. \[J=F*t\] \[J=\Delta P\] Impulsive forces are usually \textbf{high magnitude} and \textbf{short duration}. 

% Law of Conservation of Momentum
\subsection{Law of Conservation of Momentum}
If the resultant external force on a system is zero, then the vector sum of the momentums of the objects will remain constant. \[\sum P\textsubscript{before}=\sum P\textsubscript{after}\]

% External versus Internal Forces
\subsection{External versus internal forces}
\textbf{External forces} are forces coming from outside the system of particles whose momentum is being considered. \textbf{Internal forces} are forces arising from interaction of particles within a system. External forces change the momentum of the system while internal forces do not.

% Explosions
\subsection{Explosions}
When an object separates suddenly, as in an explosion, all forces are internal. Momentum is therefore conserved in an explosion. There is also an increase in kinetic energy in an explosion., This comes from a potential energy decrease due to chemical combustion.

% Recoil
\subsection{Recoil}
Guns and cannons \textbf{"recoil"} when fired. \textbf{This means the gun or cannon must move backwards as it propels the projectile forward.} The recoil is the result of action0reaction force pairs, and is entirely due to internal forces. As the gasses from the gunpowder explosion expand, they push the projectile forwards and the gun or cannon backwards. 