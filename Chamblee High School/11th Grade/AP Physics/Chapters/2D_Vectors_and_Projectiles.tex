% 2-D Vectors and Projectiles
\section{2-D Vectors and Projectiles}
	
% Overview
\subsection{Overview}
2-D Dimensional Motion is motion that occurs with both x and y components. Each dimension of the motion can obey different equations of motion.
	
% How to Solve 2-D Problems
\subsection{How to Solve 2-D Problems}
\begin{enumerate}
	\item Resolve all the vectors into x and y components
	\item Work the problem as two one-dimensional problems
	\item Re-combine the results for the two components at the end of the problem. 
\end{enumerate}
	
% Projectiles Motion
\subsection{Projectile Motion}
A projectile is something is fired, throw, shot, or hurled near the earth's surface. Horizontal velocity is constant, vertical velocity is accelerated, and air resistance is ignored. 
	
% 1-D Projectiles
\subsection{1-Dimensional Projectiles}
A 1-Dimensional projectile is a projectile that movies in a vertical direction only and is subject to acceleration by gravity. Here you can only calculate vertical motion because the motion has no horizontal component.
	
% 2-D Projectiles
\subsection{2-Dimensional Projectiles}
A 2-Dimensional projectile is a projectile that moves both horizontally and vertically and is subject to acceleration by gravity in the vertical direction. Here you can calculate both vertical and horizontal motion.
	
% Horizontal Component of Velocity
\subsection{Horizontal Component of Velocity}
The horizontal component of velocity is always constant, not accelerated, and not influenced by gravity. It follows this equation: \[x=V_{ox}t\]
	
% Vertical Component of Velocity
\subsection{Vertical Component of Velocity}
The vertical component of velocity undergoes accelerated motion and is accelerated by gravity.
\[V_y=V_{oy}+gt\]
\[y=y_o+V_{oy}+\frac{1}{2}gt^2\]
\[V_y^2=V_{oy}^2+2g(y-y_o)\]
	
% Launch Angle
\subsection{Launch Angle}
Launch angle is the angle at which a projectile is launched. The launch angle determines what the trajectory for he projectile will be. Launch angles can range from $-90^\circ$ (throwing something straight down) to $+90^\circ$ (throwing something straight down) and everything in between. A zero launch angle implies a perfectly horizontal launch.
	
% General Launch Angle
\subsection{General Launch Angle}
Projectile motion is more complicated when the launch angle is not straight up or down or perfectly horizontal. You must being these types of problems by resolving the velocity vector into its components. Then use speed and the launch angle to find the horizontal and vertical velocity components. \[V_{oy}=V_o\sin\theta\]
	
Use speed and the launch angle to find both the horizontal and vertical components. 
\[V_{ox}=v_o\cos\theta\]
\[V_{oy}=v_o\sin\theta\]
	
% Projectiles launched over level ground
\subsection{Projectiles launched over level ground}
These projectiles have highly symmetric characteristics of motion. It is handy to know these characteristics, since a knowledge of the symmetry can help in working problems and predicting the motion.
	
% Trajectory of a 2-D Projectile
\subsection{Trajectory of a 2-D Projectile}
The trajectory is the path traveled by any projectile. It is plotted on an x-y graph. Mathematically, the path is defined by a parabola. For a projectiles launched over level ground, the symmetry is apparent.
	
% Range of a 2-D Projectile
\subsection{Range of a 2-D Projectile}
The range of the projectile is how far it travels horizontally.
	
% Maximum height of a 2-D Projectile
\subsection{Maximum height of a 2-D Projectile}
The maximum height of the projectile occurs when it stops moving upward. The vertical velocity component is zero at the maximum height. For a projectile launched over level ground, the maximum height occurs halfway through the flight of the projectile. 
	
% Acceleration of a Projectile
\subsection{Acceleration of a Projectile}
Acceleration points down at $9.8m/s^2$ for the entire trajectory of all projectiles. 
	
% Velocity of a Projectile
\subsection{Velocity of a Projectile}
Velocity is tangent to the path for the entire trajectory. The velocity can be resolved into components all along its path. The vertical velocity changes while the horizontal velocity remains the same. Maximum speed is attained at the beginning and again at the end of the trajectory if the projectile is launched over level ground. Launch angle is symmetric with landing angle for a projectile over level ground.
	
% Time of flight for a Projectile
\subsection{Time of flight for a Projectile}
The projectile spends half its time traveling upward and the other half traveling down.
	
% The Range Equation
\subsection{The Range Equation}
Derivation is an important part of physics. The Range equation is in your textbook but not on your formula sheet. \[R=\frac{V_o^2\sin(2\theta)}{g}\]
\begin{itemize}
	\item $R$ = Range of the projectile fired over level ground
	\item $v_o$ = initial velocity
	\item $g$ = acceleration due to gravity
	\item $\theta$ = launch angle
\end{itemize}