% Rotational Motion
\section{Rotational Motion}

% Rotational Kinematics
\subsection{Rotational Kinematics}
In translational motion, the position is represented by a point such as x. In rotational motion, the position is represented by an angle such as $\theta$, and a radius $r$.\\

Linear displacement is represented by the vector $\Delta x$. Angular displacement is represented by $\Delta \theta$, which is not a vector, but behalves like one for small values. The right-hand rule determines direction. \\

A particle that rotates through an angle $\Delta\theta$ also translates through a distance $s$, which is the length of the arc defining its path. This distance $s$ is related to the angular displacement $\Delta\theta$ by the equation $s=r\Delta\theta$. \\

The instantaneous velocity has magnitude $v_T=\frac{ds}{dt}$ and is tangent to the circle. The same particle rotates with an angular velocity with $\omega=\frac{d\theta}{dt}$. The direction of the angular velocity is given by the right-hand rule. Tangential and angular speeds are related by the equation $v=r\omega$.\\

% Acceleration
\subsection{Acceleration}
Tangential acceleration is given by $a_T=\frac{dv_t}{dt}$. This acceleration is parallel or antiparallel to the velocity.\\

Angular acceleration of this particle is given by $\alpha=\frac{d\omega}{dt}$. Angular acceleration is parallel or anti-parallel to the angular velocity.\\

Tangental and angular acceleration are related by the equation $a=r\alpha$.\\

Centripetal acceleration always points to the center of the circle and is $a_c=\frac{v^2}{r}$ or $a_c=r\omega^2$.

% Kinematic Equations
\subsection{Kinematic Equations}
\[\omega=\alpha t\]
\[\theta=\theta_o+\omega_ot+\frac{\alpha}{t^2}\]
\[\omega^2=\omega^2_o+2\alpha(\omega-\omega_o)\]

% Rotational Energetics
\subsection{Rotational Energetics}


\subsection{Inertia and Rotational Inertia}
In linear motion, inertia is equivalent to mass. Rotating systems have "rotational inertia."
\[I=\sum mr^2\]
\begin{itemize}
	\item I: rotational inertia ($kg m^2$)
	\item m: mass ($kg$)
	\item r: radius of rotation ($m$)
\end{itemize}

% Kinetic Energy
\subsection{Kinetic Energy}
\[K_{\text{trans}}=\frac{1}{2}mv^2\]
\[K_{\text{rot}}=\frac{1}{2}I\omega^2\]
\[K_{\text{rot}}=\frac{1}{2}mv^2+\frac{1}{2}I\omega^2\]

% Rotational Inertia Calculations
\subsection{Rotational Inertia Calculations}
\begin{center}
		\item System of Particles: $I=\sum mr^2$
		\item Solid object: $I=\int r^2 dm$
		\item Parallel Axis Theorem: $I= I_{\text{cm}}+m h^2$
\end{center}