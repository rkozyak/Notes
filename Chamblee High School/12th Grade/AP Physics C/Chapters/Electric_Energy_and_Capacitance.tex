% Electric Energy and Capacitance
\section{Electric Energy and Capacitance}

% Electric Potential Energy
\subsection{Electric Potential Energy}
The electrostatic force is a conservative force and it is possible to define an electrical potential energy function with this force. Woke done by a conservative force ie equal to the negative change in potential energy.

% Work and Potential Energy
\subsection{Work and Potential Energy}
There is a uniform field between two plates. As a positive charge charge moves from $A$ (high PE) to $B$ (low PE), work is done on it.
\[W=Fd=qE_x(x_f-x_i)\]
\[\Delta PE=-W=-qE_x (x_f-x_i)\]

% Potential Difference
\subsection{Potential Difference}
The potential difference between points $A$ and $B$ is defined as the change in the potential energy of a charge $q$ that moved from $A$ to $B$ divided by the size of the charge. Potential difference is not the same as potential energy. \\
\[\Delta V=V_B-V_a=\frac{\Delta PE}{q}\]
\[\Delta PE = q\Delta V\]
Both electric potential energy and potential difference are scalar quantities. The units of potential difference is $V = \frac{J}{C}$.\\

A special formula can be used when there is a uniform electric field:
\[\Delta V= V_b-V_a=-E\Delta x\]

% Energy and Charge Movements
\subsection{Energy and Charge Movements}
A positive charge gains electrical potential energy when it is moved in a direction opposite the electric field. A negative charge loses electric potential energy when it moves in the direction opposite the electric field.\\

If a charge is released in the electric field, it experiences a force and accelerates and gains kinetic energy. As it gains kinetic energy, it loses an equal amount of electrical potential energy.\\

% Summary of Positive Charge Movements and Energy
\subsection{Summary of Positive/Negative Charge Movements and Energy}
When a positive charge is placed in an electric field
\begin{itemize}
	\item it moves in the direction of the field:
	\item it moves from a point of higher potential to a point of lower potential
	\item its electric potential energy decreases
	\item its kinetic energy increases\\
\end{itemize}

When a negative charge is placed in an electric field:
\begin{itemize}
	\item it moves opposite to the direction of the field
	\item it moves from a point of power potential to a point of higher potential
	\item its electrical energy increases
	\item its kinetic energy decreases
	\item Work had to be done on the charge for it to move from point A to point B
\end{itemize}

% Electric Potential of a Point Charge
\subsection{Electric Potential of a Point Charge}
The point of zero electric charge is taken to be at an infinite distance from the charge. The potential created by a point charge $q$ at any distance $r$ from the charge is:
\[V=k_e\frac{q}{r}\]
A potential exists at some point in space whether or not there is a test charge at that point.

% Electric Field and Electric Potential Depend on Distance
\subsection{Electric Field and Electric Potential Depend on Distance}
The electric field is proportional to: \[\frac{1}{r^2}\]
The electric potential is proportional to: \[\frac{1}{r}\]

% Electric Potential of Multiple Point Charges
\subsection{Electric Potential of Multiple Point Charges}
The superposition principle still applies so the total electric potential at some point $P$ due to serval point charges is the algebraic sum of the electric potentials due to the sum of the individual charges.

\subsection{Electric Potential Energy of Two Charges}
$V_1$ is the electric potential due to $q_1$ at some point $P$. The work required to bring $q_2$ from infinity to $pP$ without acceleration is $q_2V_1$. This work is equal to the potential energy of the two particle systems is:
\[PE=q_2V_1=k_e\frac{q_1q_2}{r}\]

% Notes about Electric Potential Energy of two charges
\subsection{Notes about Electric Potential Energy of two charges}
If the charges have the same sing, PE is positive and work must be done to force the two charges near each other and the like charges would repel.\\

If the charges have opposite signs, PE is negative and the force would be attractive. Work must be done to hold back the unlike charges from accelerating as they are brought close together. 

% Potentials and Charged Conductors
\subsection{Potentials and Charged Conductors}
Because no work is required to move a charge between to points that are at the same electric $W=0$ when $V_a=V_b$. All points on the surface of. charged conductor in electrostatic equilibrium are at the same potential. Therefore, the electric potential is a constant everywhere on the surface of a charged conductor in equilibrium.

% Conductors in Equilibrium
\subsection{Conductors in Equilibrium}
The conductor has an excess of a positive charge and all of the charge resides at the surface. $E=0$ inside the conductor. The electric field just outside the conductor is perpendicular to the surface. The potential is constant everywhere on the surface of the conductor. The potential everywhere inside is constant and equal to tits value on the surface.

% Electron Volt
\subsection{Electron Volt}
The electron volt (eV) is defined as the energy that an electron gains when accelerated through a potential difference of $1$ v.
\[1\text{ eV }=1.6*10^{-19}\text{ J }\]

% Equipotential Surface
\subsection{Equipotential Surface}
An equipotential surface is a surface on which all points are at the same potential and no work is required to move a charge at a constant speed on an equipotential surface and the electric field at every point on the surface is perpendicular to the surface.

% Equipotentials and Electric Field Lines - Positive Charge
\subsection{Equipotentials and Electric Field Lines - Positive Charge}
The equipotentials for a point charge are a family of spheres centered on the point charge and the field lines are perpendicular to the electric potential at all points.

% Equipotentials and Electric Field lInes - Dipole
\subsection{Equipotentials and Electric Field lInes - Dipole}
Equipotential lines are shown in blue and electric field lines are shown in red. The field lines are perpendicular to the equipotential lines at all points.

% Electrical Potential Energy
\subsection{Electrical Potential Energy}
\[U=qV=\frac{kq_1q_2}{r}=\frac{q_1q_2}{4\pi\epsilon_0r}\]
This formula works for the potential energy of two point charges. For more than two charges, you must add the potential energy contribution due to each pair.
