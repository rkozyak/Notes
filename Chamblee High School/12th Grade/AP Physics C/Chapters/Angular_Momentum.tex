% Angular Momentum
\section{Angular Momentum}

% Angular Momentum
\subsection{Angular Momentum for Particle}
Angular momentum is a quantity that tells us how hard it is to change the rotational motion of a particular spinning body. Objects with lots of angular momentum are hard to stop spinning or to turn and have great orientational stability. 

For a single particle:
\[L= r \text{x} p\]
\begin{itemize}
	\item L: angular momentum for a single Particle
	\item r: distance from particle to point of rotation
	\item p: linear momentum
\end{itemize}

% Angular Momentum for solid Object
\subsection{Angular Momentum for solid Object}
For a solid object, angular momentum is analogous to linear momentum of a solid object
\[P=mv\]
\begin{itemize}
	\item Replace momentum with angular momentum
	\item Replace mass with rotational inertia
	\item Replace velocity with angular velocity
\end{itemize}

\[L=I\omega\]

% Law of Conservation of Angular Momentum
\subsection{Law of Conservation of Angular Momentum}
The momentum of a system will not change unless an external force is applied. Angular momentum of a system will also not change unless an external torque is applied.

% Angular Momentum and Torque
\subsection{Angular Momentum and Torque}
\[F=\frac{dP}{dt}\]
\[\tau=\frac{dL}{dt}\]

If there is an unbalanced torque, then torque changes L with respect to time. Torque increases angular momentum when the two vectors are parallel and decreases when they are anti-parallel.

Torque changes the direction of the angular momentum vector in all other situations. This results in what is called the \textbf{percussion} of spinning tops.