% Vectors in 2-D motion
\section{Vectors in 2-D Motion}

% Scalars vs Vectors
\subsection{Scalars vs Vectors}
\begin{itemize}
	\item \textbf{Scalars} have magnitude only. (distance, speed, time, mass)
	\item \textbf{Vectors} have both magnitude and direction (displacement, velocity, acceleration)
	\item \textbf{Vectors} have both magnitude and direction (displacement, velocity, acceleration)
	\item The direction of a vector is represented by the direction in which the ray points and is typically given by an angle.
	\item The magnitude of a vector is the size of whatever the vector represents. It is represented by the length of the vector and is often represented as |A|.
	\item Equal vectors have the same length and direction and represent the same quantity (such as force or velocity).
	\item Inverse vectors have the same length but opposite direction.
	\item Vectors are added graphically together head-to-tail. The sum is called the resultant. The inverse of the sum is called the equilibrant. 
\end{itemize}


% Unit Vectors
\subsection{Unit Vectors}
Unit vectors are quantities that specify direction only. They have a magnitude of exactly one, and typically point in the x, y, or z directions. \\

%\^{i} points in the x direction \\

%\^{j} points in the y direction \\

%\^{k} points in the z direction \\

\begin{table}[h]
\centering
\begin{tabular}{ccc}
\^i points in the x direction & \^j points in the y direction & \^k points in the z direction
\end{tabular}
\end{table}