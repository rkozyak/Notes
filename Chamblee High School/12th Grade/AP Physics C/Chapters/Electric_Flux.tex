% Electric Flux
\section{Electric Flux}

% Flux
\subsection{Flux}
Flux means "flow". You can increase flux in three ways:
\begin{itemize}
	\item increase the field
	\item increase the area of the loop
	\item make sure the loop is appropriately angled
\end{itemize}

% The Area Vector
\subsection{The Area Vector}
The area vector is defined as a vector perpendicular to a surface with a magnitude equal to the scalar area of the surface. Flux is proportional to the field vector magnitude, area vector magnitude, and the cosine of the angle between them and the units is Nm$^2$/C. For a closed shape each side has its own area vector.
\[\Phi = v \cdot A\]
\[\Phi = E \cdot A\]

% Calculation of flux over a closed surface in a vector field
\subsection{Calculation of flux over a closed surface in a vector field}
If there is a "source" of the vector field in the closed shape, the flux over the surface is positive. If there is a "sink" of the vector field in the closed shape, the flux over the surface is negative. 
For a general vector:
\[\Phi = \oint v\cdot dA\]
For an electric field:
\[\Phi = \oint E \cdot dA\]

% Gauss's Law of Electricity
\subsection{Gauss's Law of Electricity}
\[q=\epsilon_o\Phi_e\]
$q$ is the net charge enclosed inside a given gaussian surface and is the sum of all the + and - charges. $\epsilon_o$ is the electrical permittivity of free space. Other forms include:
\[q=\epsilon_o\Phi_e\]
\[\oint E\cdot dA = \frac{q}{\epsilon_o}\]
\[\nabla \cdot E = \frac{\rho}{\epsilon_o}\]
\[\rho = \frac{q}{v}\]

% Choosing Gaussian Surfaces
\subsection{Choosing Gaussian Surfaces}
Conditions when choosing a surface:
\begin{itemize}
	\item The value of the electric field can be argued by symmetry to be constant over the surface.
	\item The dot product of $E\cdot dA=E(dA)$ because $E$ and $A$ are in the same direction.
	\item The dot product of $E\cdot dA = 0$ because $E$ and $a$ are perpendicular.
	\item The filed can be argued to be zero over the surface. 
\end{itemize}

