% Potential Energy
\section{Potential Energy}

% Types of Forces
\subsection{Types of Forces}
\noindent Conservative foces:
\begin{itemize}
	\item Work in moving an object is path independent
	\item Work in moving an object along a closed path is zero.
	\item Work is directly related to a negative change in potential energy.
\end{itemize}

\noindent Non-Conservative Foces:
\begin{itemize}
	\item Work is path dependent
	\item Work along a closed path is not zero
	\item Work may be relayed to a change is mechanical energy or thermal energy.
\end{itemize}

% Potential Energy
\subsection{Potential Energy}
A type of mechanical energy possessed by an object by virtue of its position or configuration. It is represent  by the latter $U$. The work done by conservative forces is the negative of the potential energy change.
\[W=-\Delta U\]

% Gravitational Potential Energy
\subsection{Gravitational Potential Energy}
The change in gravitational potential energy is the negative of the work done by gravitational force on an object when it is moved. For objects near the earth's surface, the gravitational pull of the earth is roughly constant, so the force necessary to lift an object at constant velocity is equal to the weight, so we can say:
\[\Delta U_g=-W_g=mgh\]

% Spring Potential Energy
\subsection{Spring Potential Energy, $U_s$}
Springs obey Hooke's Law and unlike gravitational potential energy, we know where the zero potential energy point is for a spring.
\[F_s(x)=-kx\]
\[W_s=\int F_s(x)dx=-k\int xdx =-\frac{1}{2}kx^2\]
\[U_s=\frac{1}{2}kx^2\]