% Charge and Polarization
\section{Charge and Polarization}

% The Atom
\subsection{The Atom}
The atom has positive charge in the nucleus, which is located in the protons. The positive charge cannot move from the atom unless there is a nuclear reaction. \\

The atom has negative charge in the election cloud on the outside of the atom. Electrons can move from the atom to another atom without all that much difficulty.


% Charge
\subsection{Charge}
Charge comes in two forms, which Ben Franklin designated as positive (+) and negative (-). Charge is quantized and the smallest possible stable charge, which we designate as $e$, is the magnitude of the charge on 1 electron or proton. $e$ is referred to as the electron

% Coulomb's Law
\subsection{Coulomb's Law}
Coulomb's Law states that the force between any two objects depends on their charges, the constant $k$, and the distance between the objects. 

\[F=\frac{k|q_1| |q_2|}{r^2}\]
\[k=9\text{x}10^9 \text{Nm}^2/\text{c}^2\]
\[k=\frac{1}{4\pi\epsilon _o}\]

% Superposition
\subsection{Superposition}
When one or more charge contributes to the electric field, the resultant electric field is the vector sum of the electric fields. 


% The Electric Field
\subsection{The Electric Field}
The presence of + or - charge modifiers empty space. This enables the electrical force to at on charged particles without actually touching them. The electric field is created in the space around a charged particle or a configuration of charges.\\

If a charged particle is placed in an electric field created by other charges, it will experience a force as a result of the field.\\

Why use fields?
\begin{itemize}
	\item Forces exist only when two or more particles are present
	\item Fields exist even if no force is present
	\item The field of one particle only can be calculated
\end{itemize}

The lines of force indicate the direction of the force on a positive charge. These lines of force point away from positive charges and into negative charges.

% Field Vectors from Field Lines
\subsection{Field Vectors from Field Lines}
The electric field at a given point is not the field line itself, but can be determined by taking the tangent to the field at any point. 

% Force from Electric Field
\subsection{Force from Electric Field}
\[F=E q\]
\begin{itemize}
	\item F: Force (N)
	\item E: Electric Field (N/C)
	\item q: Charge (C)
\end{itemize}

\subsection{For Spherical Electric Fields}
The electric field surrounding a point charge or a spherical charge can be calculated by:
\[E=k\frac{q}{r^2}\]
\begin{itemize}
	\item E: Electric Field (N/C)
	\item k: 8.99 x $10^9$
	\item q: Charge (C)
	\item r: Distance from center of charge q (m)
\end{itemize}