% Charge Distribution
\section{Charge Distribution}

% Limitations of Coulomb's Law
\subsection{Limitations of Coulomb's Law}
Coulomb's Law equations for Force and field can only be used directly for point changes or spherically symmetric charges. For more complicated "Continuous charge distributions" we need to need to integrate. 

% Linear Charge Distribution
\subsection{Linear Charge Distribution}
When charge resides on a long thin object such as a wire or a ring, we can that a linear charge distribution.
\[\lambda =\frac{Q}{L} = \frac{dQ}{dL}\]

% Surface Charge Distribution
\subsection{Surface Charge Distribution}
When change resides on larger surface, we can it a surface charge distribution. It is sometimes convenient for us to define a surface charge density, $\sigma$, which is charge per unit area.
\[\sigma=\frac{Q}{A}=\frac{dQ}{dA}\]

% Volume Charge Distribution
\subsection{Volume Charge Distribution}
% When charge resides distributed within a solid object, we have a
\[ = \frac{Q}{V} = \frac{dQ}{dv}\]

% General Procedure
\subsection{General Procedure}
\[E=\int dE = \int \frac{kdq}{r^2}\]
You need to integrate of a spatial variable and find a common variable that $r$ and $dq$ both depend on and find the appropriate limits to the integral.

% Motion of Charged Particles in Electric Fields
\subsection{Motion of Charged Particles in Electric Fields}
\[F=qE=ma\]
The motion is not unlike projective motion and the electric field is constant so kinematic equations can be employed.

% Work Done by Electric Field
\subsection{Work Done by Electric Field}
\[W=\int_A^BF\cdot r= q \int_A^B \]
\[\Delta U = -q\int E\cdot dr\]
\[\Delta V = -\int E \cdot dr\]
\[V=\frac{kq}{r}\]

\subsection{Point Charge Potential Derivation}