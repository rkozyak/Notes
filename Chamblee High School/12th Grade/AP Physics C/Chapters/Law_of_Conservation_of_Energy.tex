% Law of Conservation of Energy
\section{Law of Conservation of Energy}
If a system is isolated and the boundary allows no exchange. with the environment then: $E=U+K+E_{\text{int}}=\text{Constant}$
\[E=U+K=C\]
\[\Delta E= \Delta U + \Delta K =0\]

\begin{center}
For Gravity:
\[\Delta U_g + mgh_f-mgh_i\]
\[\Delta K =\frac{1}{2}mv_f^2-\frac{1}{2}mv_i^2\]
\\
For Springs:
\[\Delta U_s = \frac{1}{2}kx_f^2-\frac{1}{2}kx^2_i\]
\[\Delta K = \frac{1}{2}mv_f^2-\frac{1}{2}mv_i^2\]
\end{center}

% Pendulum Energy
\subsection{Pendulum Energy}
For any two points in the pendulum's swing:
\[\frac{1}{2}mv_1^2+mgh_1=\frac{1}{2}mv^2_2+mgh_2\]

% Spring Energy
\subsection{Spring Energy}
\[\frac{1}{2}kx_1^2+\frac{1}{2}mv_1^2=\frac{1}{2}kx_2^2+\frac{1}{2}mv_2^2\]

% Forces and Potential Energy
\subsection{Forces and Potential Energy}
In order to discuss the relationship between displacment and forces, we need to know a couple of equations:
\[W=\int F(x)dx=-\int dU=\Delta U\]
\[\int dU= \int F(x)dx\]
\[F(x)=-dU(x)/dx\]

% Stable, Unstable, and Neutral Equilibrium
In stable equilibrium, the forces are trying to bring the object back to equilibrium. If the forces are trying to move the object away from equilibrium then it at unstable equilibrium. When the system is displaced from equilibrium and it just stays there, it is at neutral equilibrium. 