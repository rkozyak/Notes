% Newton's Laws
\section{Newton's Laws}

% Force
\subsection{Force}
Force is a push or pull on an object. Unbalanced forces cause an object to accelerate. (ie speed up, slow down, or change directions). Force is also a vector

% Types of Forces
\subsection{Types of Forces}
\begin{center}
	\begin{enumerate}
		\item Contact forces: involves contact between bodies (ie normal/friction)
		\item Field forces: act without necessity of contact (ie gravity/electromagnetic)
	\end{enumerate}
\end{center}


% Forces and Equilibrium
\subsection{Forces and Equilibrium}
If the net fore on a body is zero, it is in \textbf{equilibrium}. An object in equilibrium may be moving relative to us (dynamic equilibrium) or may appear to be at rest (static equilibrium).

% Newton's First Law - The Law of Inertia
\subsection{Newton's First Law: The Law of Inertia}
\begin{center}
	\textbf{A body in motion stays in motion in a straight line unless acted upon by an external force.} 
\end{center}

This law is commonly applied to the horizontal component of velocity, which is assumed not to change during the flight of a projectile. 

% Newton's Second Law - F=MA
\subsection{Newton's Second Law: $F=m*a$}
\[\sum F=m*a\]
A body accelerates when acted upon by a net external force. The acceleration is proportional to the next force and is in the direction which the net force acts. This law is commonly applied to the vertical component of velocity in projectiles. 

% Newton's Third Law
\subsection{Newton's Third Law}
For every action there exists an equal and opposite reaction. If A exerts a force F on B, then B exerts a force of -F on A

% Commonly Confused Terms
\subsection{Commonly Confused Terms}
\begin{enumerate}
	\item \textbf{Inertia} or the resistance of an object to being accelerated
	\item \textbf{Mass:} the same thing as inertia (to a physicist)
	\item \textbf{Weight:} gravitational attraction.
\end{enumerate}
\begin{center}
	inertia = mass $\alpha$ weight
\end{center}

% Normal Force
\subsection{Normal Force}
Normal force is the force that keeps one object from invading another object. Our weight is the force of attraction of our body for the center of the planet.
\[N=mg*\cos\theta\]

% Tension
\subsection{Tension}
Tension is a pulling force. Generally it exists in a ripe, spring, or cable. It arises at the molecular level, when a rope string, or cable resists being pulled apart.

The horizontal and vertical components of the tension are equal to zero if the system is not accelerating. 

% Atwood Machine
\subsection{Atwood Machine}
An Atwood machine is a device used for measuring g. If m1 and m2 are nearly the same, it slows down freefall such that acceleration can be measured. Then g can be measured. 