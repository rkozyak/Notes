% Capacitance
\section{Capacitance}
\subsection{Capacitors}
A capacitor is a device used in a variety of an electric circuits. The capacitance, $C$, of a capacitor is defined as the ratio of the magnitude of the charge on either conductor (plate) to the magnitude of the potential difference between the conductors (plates). It's units is Farad (F) and $1F=\frac{C}{V}$.
\[C=\frac{Q}{\Delta V}\]

% Parallel-Plate Capacitor
\subsection{Parallel-Plate Capacitor}
The capacitance of a device depends on the geometric arrangement of the conductors. For a parallel-plate capacitor whose plates are separated:
\[C=\frac{k\epsilon_o A}{d}\]

Capacitors consist of two parallel plates with an area $A$ and are separated by a distance $d$ and both plates carry equal and opposite charges. When connected to a battery, charge is pulled off one plate and transferred to the other plate and the transfer stopes when $\Delta V_{\text{cap}}=\Delta V_{\text{battery}}$

% Electric Field in a Parallel-Plate Capacitor
\subsection{Electric Field in a Parallel-Plate Capacitor}
The electric field between the plates is uniform near the center and nonuniform near the edges. The filed may be taken as constant thought the region between the plates.

% Capacitors in Circuits
\subsection{Capacitors in Circuits}
A circuit is a collection of objects usually containing a source of electric energy connected to elements that convert it to other forms. A circuit diagram can be used to show the path of the real circuit. 

% Capacitors in Parallel
\subsection{Capacitors in Parallel}
When capacitors are first connected in the circuit, electrons are transferred from the left plates through the battery to the right plate, leaving the left plate positively charged and the right plate negatively charged.

The total charge is equal to the sum of the charges on the capacitors:
\[Q_{\text{total}}=Q_1+Q_2\]

The potential difference across the capacitors is the same and each is equal to the voltage of the battery. The capacitors can also be replaced with one capacitor with a capacitance of $C_{\text{eq}}$. The equivalent capacitor must have exactly the same external effect on the circuit as the original capacitors.

% Capacitors in Series
\subsection{Capacitors in Series}
When a battery is connected to the circuit, electrons are transferred from the left plate of $C_1$ to the right plate of $C_2$ through the battery. As this negative charge accumulates on thr right plate of $C_2$m ab equivalent amount of negative charge is removed from the left plate of $C_2$.
\[\frac{1}{C_{\text{eq}}}=\frac{1}{C_1}+\frac{1}{C_2}\]
\[C=\frac{Q}{\Delta V}\]
\[C_{\text{eq}}=C_1+C_2\]
Capacitors in parallel have the same voltage differences as does the equivalent capacitance. Capacitors in series all have the same charge as does their equivalent capacitance.
\[\text{Energy Stored} = \frac{1}{2}Q\Delta V\]
\[\text{Energy} = \frac{1}{2}Q\Delta V=\frac{1}{2}C\Delta V^2=\frac{Q^2}{2C}\]

% Capacitors with Dielectrics
\subsection{Capacitors with Dielectrics}
A dielectric is an insulating material that, when placed between the plates of a capacitor, increase the capacitance. Dielectrics include rubber, plastic, or waxed paper.
\[C=\kappa C_o=\kappa e_o(A/D)\]










