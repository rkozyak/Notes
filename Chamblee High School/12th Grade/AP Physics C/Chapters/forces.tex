% Foces
\section{Forces}

% Variable Forces
\subsection{Variable Forces}
Forces can vary with time, velocity and with position.

\subsection{Calculus Concepts}
\begin{center}
Differentiation gives you the tangent of a function:\\ Position $\rightarrow$ Velocity $\rightarrow$ Acceleration
\end{center}

\begin{center}
Integration gives you the area under a curve:\\ Acceleration $\rightarrow$ Velocity $\rightarrow$ Position
\end{center}

\begin{center}
If $a(t)=t^n$ then $\int t^n dt = \frac{t^{n+1}}{n+1}+C$
\end{center}

% Drag Forces
\subsection{Drag Forces}
Drag forces slow an object down as it passed through a fluid. They act in the opposite direction to velocity, are functions of velocity, and impose terminal velocity. 

% Drag as a Function of Velocity
\subsection{Drag as a Function of Velocity}
\[f_D=bv+cv^2\]
$b$ and $c$ depend upon shape and size of the object and the properties of fluid. $b$ is important at low velocity while $c$ is important at high velocity.

% Drag Forces in Free Fall
\subsection{Drag Forces in Free Fall}
When $f_D$ equals $mg$, terminal velocity has been reached.\\
\begin{center}	
For Slow moving objects: $F_D=BV$\\
For fast moving objects: $F_D=cv^2$\\
\end{center}

\[c=\frac{1}{2}D*p*A\]
Where: $D$ = drag coefficient, p = density of fluid, and A = cross-sectional area