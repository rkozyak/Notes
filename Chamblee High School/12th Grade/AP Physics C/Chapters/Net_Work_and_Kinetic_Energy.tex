% Net Work and Kinetic Energy
\section{Net Work and Kinetic Energy}

\subsection{Net Work or Total Work}
An object can be subject to many forces at the same time, and if the object is moving the work done on each force and be individually determined. At the same time one force does positive work on the object, another force may be doing negative work, and yet another force may be doing no work at all. The net work, or total, work done on the object is the scalar sum of the work done on an object by all forces acting upon the object.
\[W_{net}=\sum W_i\]
\[W_{\text{net}}=\sum W_i\]


\subsection{The Work Energy Theorem}
\[W_{\text{net}=\Delta K}\]
When net work due to all forces acting upon an object is positive, the kinetic energy of the object will increase. When the net work due to all forces acting upon an object is negative, the kinetic energy of the object will increase. When there is no net work acting upon an object, the kinetic energy of the object will be unchanged. 

\subsection{Kinetic Energy}
Kinetic energy is one form of mechanical energy, which is energy we can easily see and characterize. Kinetic energy is due to the motion of an object.
\[K=\frac{1}{2}mv^2\]
\[K=\frac{1}{2}m v\cdot v\]

% Power
\subsection{Power}
Power is the rate of which work is done, No matter how fast we get up the stairs, our work is the same. If you run then your power is higher than if you walked.
\[P_{\text{ave}}=\frac{W}{t}\]
\[P_{\text{inst}} = \frac{dW}{dt}\]
\[\]

% Units of Power
\subsection{Units of Power}
\begin{enumerate}
	\item Watt= J/s
	\item ft lb/s
	\item horsepower (1 HP = 747 Watts)
\end{enumerate}