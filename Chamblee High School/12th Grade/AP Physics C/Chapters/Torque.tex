\section{Torque}

% Torque as a vector
\subsection{Torque as a vector}
Torque is the rotational Analog of force that causes rotation to begin. The magnitude of the torque is proportional to that of the force and moment arm.
\[\tau = Fr \sin \theta\]
\[\tau = r x F\]


% Equilibrium
\subsection{Equilibrium}
Equilibrium occurs when there is no net force and no net torque on a system. Static equilibrium occurs when nothing in the system is moving or rotating your reference frame. Dynamic equilibrium occurs when the system is translating at constant velocity and/or rotation at constant rotational velocity.
\[\sum F=0\]
\[\sum \tau = 0\]

% Torque and Newton's 2nd Law
\subsection{Torque and Newton's $2^{\text{nd}}$ Law}
\[\sum\tau=I\alpha\]



% Work in Rotating Systems
\subsection{Work in Rotating Systems}
\[W=F\cdot \Delta r\]
\[W_{\text{rot}}=\tau\cdot\Delta\theta\]
\[W_{\text{net}} = \Delta K\]
\[W_c=-\Delta U\]
\[W=\Delta E\]

% Power in Rotating System
\subsection{Power in Rotating Systems}
\[P=\frac{dW}{dt}\]
\[P=F\cdot v\]
\[P_{\text{rot}}=\tau\cdot\omega\]

% Conservation of Energy
\subsection{Conservation of Energy}
\[E_{\text{rot}} = U+K=\text{Constant}\]


% Rolling without Slipping
\subsection{Rolling without Slipping}
Total kinetic energy of a body is the sum of the translational and rotational kinetic energies.
\[K=\frac{1}{2}Mv^2+\frac{1}{2}I\omega^2\]
When a body is rolling without slipping, another equation holds true:
\[V_{\text{cm}}=\omega\tau\]
Therefore, this equation can be combined with the first one to create the two following equations:
\[K=\frac{1}{2}M^2+\frac{1}{2}I_{\text{cm}}\frac{v^2}{R^2}\]
\[K=\frac{1}{2}m\omega^2 R^2+\frac{1}{2}I_{\text{cm}}\omega^2\]