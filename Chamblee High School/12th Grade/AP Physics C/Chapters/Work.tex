% Introduction to Work
\section{Introduction to Work}

% Energy and Work
\subsection{Energy and Work}
\begin{itemize}
	\item A body experiences a change in energy when one of more forces do work on it. 
	\item A force does positive work on a body when the force and the displacement are least partially aligned. 
	\item Maximum positive work is done when a force and displacement are exactly the same direction. 
	\item If a force causes no displacement, it does no work (ie normal forces or centripetal forces). 
	\item Forces can do negative work if they are pointed opposite the direction of the displacement.
\end{itemize}

% Calculating Work
\subsection{Calculating Work}
Work, which is a scalar resulting from the interaction of two vectors, is the dot product of force and displacement.
\[W=F\cdot r\]
\[W=Fr \cos \theta\]
\[W=F_xr_x+F_yr_y+F_zr_z\]

% Units of Work
\subsection{Units of Work}
\begin{itemize}
	\item Si System: Joule (N m)
	\item British System: foot-pound
	\item cgs System: erg (dyne-cm)
	\item Atomic Level: electron volt
\end{itemize}

% Work in a Pulley System
\subsection{Work in a Pulley System}
A pulley system, which has at least one pulley attached to the load, can be used to reduce the force necessary to lift a load. Amount of work done in lifting the load is not changed. The distance the force is applied over is increased, thus the force is reduced since $W=Fd$.

% Work and Variable Forces
\subsection{Work and Variable Forces}
\begin{itemize}
	\item For constant forces: $W=F\cdot r$
	\item For variable forces, you can't move far until the force changes. The force is only constant over an infinitesimal displacement: $dW=F\cdot dr$
	\item To calculate work for a larger displacement you have to take an integral: $W=\int dW=\int F\cdot dr$
\end{itemize}

The area under the curve of a graph of force $vs$ displacement gives the work done by the force.
\[W=\int^{x_b}_{x_a}F(x) dx\]